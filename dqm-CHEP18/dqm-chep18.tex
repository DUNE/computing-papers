%%%%%%%%%%%%%%%%%%%%%%% file template.tex %%%%%%%%%%%%%%%%%%%%%%%%%
%
% This is a template file for Web of Conferences Journal
%
% Copy it to a new file with a new name and use it as the basis
% for your article
%
%%%%%%%%%%%%%%%%%%%%%%%%%% EDP Science %%%%%%%%%%%%%%%%%%%%%%%%%%%%
%
%%%\documentclass[option]{webofc}
%%% "twocolumn" for typesetting an article in two columns format (default one column)
%
\documentclass{webofc}
\usepackage[varg]{txfonts}   % Web of Conferences font
%
% Put here some packages required or/and some personal commands

\usepackage{xspace}
\usepackage{tabularx}

\newcommand{\pd}{protoDUNE\xspace}
\newcommand{\filesize}{8\,GB\xspace}


%
%
\begin{document}
%
\title{The Prompt Processing System and Data Quality Monitoring in the \pd-SP experiment}
%
%%%\subtitle{Do you have a subtitle?\\ If so, write it here}
\author{ 
\lastname{Maxim Potekhin}\inst{1}\fnsep\thanks{\email{potekhin@bnl.gov}} \it{on behalf of the DUNE Collaboration}
}

\institute{Brookhaven National Laboratory, Upton, NY11973, USA}



\abstract{%
The DUNE Collaboration is conducting an experimental program named ''protoDUNE''
which includes a beam test of two large-scale  prototypes of the Liquid Argon Time
Projection Chamber located at CERN. We present the status of prompt processing
and data quality monitoring systems deployed for the single-phase version of the detector.

}
%
\maketitle
%
\section{Introduction}
\label{sec:intro}
The \pd-SP experiment is designed to study a large-scale prototype of the single-phase version of
the Liquid Argon Time Projection Chamber (LArTPC) which will eventually become
the principal element of the DUNE apparatus to be constructed at the Sanford Underground
Research Facility \cite{cdrVol1, cdrVol4}. The prototype (CERN designation NP04) is located
in the CERN North Experimental Hall and tested utilizing a dedicated beam line from the CERN SPS
accelerator complex. The run plan also includes a large number of cosmic ray triggers.
Commisioning of the detector took place in September of 2018.
%The layout of the experimental area is shown in Fig.\,\ref{fig:np02np04}, with the single-phase
%detector seen as a cubic structure (the cryostat) on the right-hand side of the diagram. 

\begin{figure}[tb]
\centering\includegraphics[width=0.7\textwidth]{figures/np04_photo_2018_v1.png}
\caption{\label{fig:np04_photo}View of the top of the \pd-SP cryostat. The direction of the particle beam is from left to right.
Readout systems are visible on top of the detector, with DAQ racks in front.}
\end{figure}


In order to provide the cosmic ray triggering capability a large array of scintillation
counters is installed outside of the cryostat. There is also an array of beam instrumentation devices
utilized in the  trigger logic and characterization of the beam. Fig.\ref{fig:np04_photo}
shows the view of the top of the cryostat with elements of the readout hardware.
% Located in the vicinity of the detectors are enclosures for the online computing infrastructure
%(including Data Acquisition, Online Buffer etc) shown schematically as yellow blocks in the upper-right
%portion of  the diagram.
There is a dedicated 20 Gb/s network connection from the experiment to the CERN central storage facilities.

The \pd-SP Data Acquisition is equipped with a capable Online Monitoring System which receives data via
the network and does processing in real time. However, certain types of processing e.g.\,application of
digital filtering to the TPC channels, counting hit candidates etc which belong to the
category of Data Quality Monitoring (DQM) require more resources than are available within
the DAQ footprint, and include jobs that take substantially longer time than processes run in the
Online Monitor. That could present a resource and data management problem for the DAQ.
The Online Monitor is a real time system
and not well suited for long term preservation and cataloging of the data produced, such as plots,
histograms, tables etc as this would substantially complicate the system.

A separate consideration is the necessity to insulate DAQ from
potential disruptions due to frequent software updates of the DQM software dictated by
the dynamic nature of the test beam experiment. 
For these reasons a separate ``\pd Prompt Processing System'' (abbreviated as \textbf{p3s}) was put
in place \cite{eps} with the following characteristics:
\begin{itemize} 

\item turnaround time on the scale of minutes (and in some cases tens of minutes)

\item lower bandwidth as compared to the Online Monitor (i.e. a small fraction of events is processed,
but in more detail)

\item extensive and scalable computing resources provided on the CERN batch facility

\item reading data from files committed to CERN central storage, with no direct coupling to the DAQ system

\item flexibility in introducing, modifying and configuring software without any risk of disruption of
critical DAQ functionality

\item substantial storage and browsing capability allowing the users to access and study
the DQM results in an efficient manner

\end{itemize}

%\begin{figure}[tb]
%\centering\includegraphics[width=1.0\textwidth]{figures/np02np04.png}
%\caption{\label{fig:np02np04}Diagram of the layout of the CERN north area with
 % location of the protoDUNE dual phase detector (center) and the single
  %phase detector (right). The direction of the particle beam is from left to right.}
%\end{figure}

\section{The protoDUNE-SP Data}
\subsection{Raw Data Parameters}
\label{sec:np04_data_rate}

The detector features the ``cold electronics'' design in which the amplifiers and digitizers
are placed within the cryostat and operate at cryogenic temperatures. There are
``Warm Interface Boards'' located outside of the cryostat which concentrate data
and transmit it to the DAQ systems via an optic fiber.
% Digital signals are fed to the Warm Interface Boards located outside of the cryostat on its top surface and then
% transmitted to the Data Acquisition System via a fiber optic line.
There are 15,360 channels read out at the 2\,MHz digitization frequency. The size of the data
read out from the detector in a single trigger cycle is approximately 230\,MB. At the nominal
trigger rate and lossless compression applied in the DAQ instantaneous data rate is about
1.5\,GB/s, and the sustained data rate to disk is 300\,MB/s.

\subsection{Data Flow and Distribution}
Principal elements of data transmission and storage in \pd are illustrated in Fig.\,\ref{fig:dataflow}.
Once the raw data are captured by the Data Acquisition System and written to the Online Buffer
located in the experimental hall  they are picked up by an instance of the \textit{Fermi FTS} system
\cite{sam,fts} which manages the transfer to the CERN distributed storage system (EOS).

EOS serves as the principal staging area  \cite{eos_role} from which the data get copied to tape storage and
are also transmitted to FNAL for offline processing by a separate instance of \textit{Fermi FTS}.
Importantly, EOS also serves to stage the input data for the prompt processing system and to
store and preserve the DQM output i.e. the various data products produced by DQM jobs. This design
ensures a large degree of independence of prompt processing and DQM from the DAQ and vice versa,
but it also introduces a latency due to the way data transfer from the Online Buffer operates. This latency
is of the scale of a few minutes and is considered acceptable in the context of DQM.

\begin{figure}[tb]
\centering\includegraphics[width=0.9\textwidth]{figures/protoDUNE_data_flow_2018_v1.png}
\caption{\label{fig:dataflow}Diagram of the protoDUNE-SP data flow}
\end{figure}


\section{Data Quality Monitoring and Prompt Processing}
\subsection{DQM Applications}
The following types of applications were included in the \pd DQM suite during the detector
commissioning:
\begin{itemize}
\item Monitoring of front-end motherboards
\item LArTPC signal processing (noise reduction, deconvolution)
\item Basic event visualization in 2D
\item Initial data preparation for an advanced 3D event display running on a separate system
\item Liquid Argon purity monitoring using cosmic $\mu$ tracks
\item Estimation of the number of hits and charge collected in each section of the LArTPC, RMS of these values
\item Estimation of signal-to-noise ratio
\end{itemize}

\noindent For a few of these metrics the system produces time-series plots in addition to the tabulated data.
The DQM applications are often called \textit{payload jobs} so as to distinguish
them to service and infrastructure jobs which exist to support the function of the overall system.


\subsection{Design of the Prompt Processing System}

Computing resources for the Data Quality Monitoring in \pd-SP are managed by
the  \textit{\pd Prompt Processing System} (the ``p3s'')
which is separate from both the DAQ and the main production system. This is similar to some High-Energy Physics
experiments which implement ``express streams'' to perform preliminary calculations on the data very soon
after it leaves the DAQ.

The  p3s allows the experiment to benefit from combination of high capacity, high
performance distributed storage (CERN EOS), CERN networking resources and its Tier-0 computing facility.
By utilizing the pilot-based approach \cite{eps} to workload management which is the cornerstone of such well-known
systems as PanDA and DIRAC \cite{panda,dirac} p3s achieves low latency of automated job submission
as compared to running jobs directly on the batch trsource (which is currently HTCondor at CERN Tier-0).
In addition, since it combines a Web service and a back-end database this approach provides excellent
monitoring capabilities.

% There are two pricipal components in the prompt processing system - a single instance
% of the p3s server and multiple pilots running on distributed resources.

%Pilot jobs (sometimes called \textit{agents}) as submitted to worker nodes which may be
%a part of a Grid site or a local batch system. Subsequently, each pilot sends a HTTP request to the p3s server
%to register and to try to obtain a description of a job that needs to be executed.
%Jobs descriptions exist as records in the p3s database as entries created independently from pilots by
%external clients (automatic or actuated by the user). The server matches a job records with a pilot and
%replies to its HTTP request with a message containing all necessary information needed for job
%execution (e.g. the path to the executable and the configuration file, additional environment variables etc).
%The pilot parses the message and initiates the job  execution. Once the job completes, the pilot repeats the cycle
%by sending anothe request to the server. The pilots are configured on the batch system to persist for a period
%of time much larger than typical execution time of the payload  jobs so the batch slot is already primed and the next
%job starts very quickly.

\subsection{Design of the DQM content service}
The results produced by the payload jobs managed by p3s can only be useful if there is an
efficient interface for the users to access these derived data. Such interface is provided in \pd-SP
by a Web service accessible through a Web browser and a few CLI utilities.
% Combination of
%functionality  p3s and DQM content servers in a single service is not the best solution since
%the workload management component (p3s) needs to maintain high availability and the extra
%load created by content requests has to be avoided.
The DQM output (the ``content'') can be stored in the system in two ways:
\begin{itemize}

\item When the basic unit of data that needs to be stored is a file e.g.\,an image in the PNG format such
file is stored on disk (EOS) and its location is recorded in a database so it can be later retrieved
and accessed through a dynamically generated Web page,  based on some selection criteria

\item When the basic unit of data is an array of numbers these are for the most part stored directly
in the database and are used in dynamically generated Web pages either in tabulated format or
to generate graphs by using Javascript

\end{itemize}



\section{Technologies and Interfaces}
\subsection{Web Services}

Both p3s and the DQM content server ares implemented as Django \cite{django} applications
with standard components such as the Apache Web server and PostgreSQL RDBMS as the backend storage
(a few other RDBMS can be used as well). All interactions with either service
are conducted via HTTP.

The p3s server is responsible for workload management by matching the pilots to the job requests,
and also provides monitoring capabilities by allowing the user to browse and navigate
tabulated data describing the state of the various objects in the system (e.g. pilots, jobs, data files etc).
The design of monitoring pages leverages the well known \textit{``django-tables2''} package which results
in a very small amout of application code. An example of monitoring pages served by p3s is
presented in Fig.\,\ref{fig:p3s_dash}.

\begin{figure}[tb]
\centering\includegraphics[width=1.0\textwidth]{figures/p3s_dash_2018_v1.png}
\caption{\label{fig:p3s_dash}The p3s dashboard}
\end{figure}

 A suite of CLI clients is provided as a component of p3s for managing pilots, jobs and
other entities. In case interaction with the server requires exhange of messages these are
formatted as JSON. 

\subsection{Interaction between p3s and the Content Server: the self-describing data}
Due to highly dynamic nature of application development in the context of the \pd beam
test it

%For tables use syntax in table~\ref{tab-1}.
%\begin{table}
%\centering
%\caption{Please write your table caption here}
%\label{tab-1}       % Give a unique label
% For LaTeX tables you can use
%\begin{tabular}{lll}
%\hline
%first & second & third  \\\hline
%number & number & number \\
%number & number & number \\\hline
%\end{tabular}
% Or use
%\vspace*{5cm}  % with the correct table height
%\end{table}
%
% BibTeX or Biber users please use (the style is already called in the class, ensure that the "woc.bst" style is in your local directory)
% \bibliography{name or your bibliography database}
%
% Non-BibTeX users please use
%
\begin{thebibliography}{99}



\bibitem{cdrVol1}
R. Acciarri et al.
\emph{Long-Baseline Neutrino Facility (LBNF) and Deep Underground Neutrino Experiment (DUNE) Conceptual Design Report Volume 1: The LBNF and DUNE Projects}.\\ ~e-Print: arXiv:\textbf{1601.05471}
 %DUNE CDR Vol 1 -- The LBNF and DUNE Projects.~e-Print: arXiv:1601.05471
%\url{http://arxiv.org/abs/1601.05471}

\bibitem{cdrVol4}
R. Acciarri et al.
\emph{Long-Baseline Neutrino Facility (LBNF) and Deep Underground Neutrino Experiment (DUNE) Conceptual Design Report, Volume 4: The DUNE Detectors at LBNF}.\\~e-Print: arXiv:\textbf{1601.02984}
%\url{http://arxiv.org/abs/1601.02984}

\bibitem{eps} M.Potekhin et al. \emph{The protoDUNE-SP experiment and its prompt
processing system}. Proceedings of Science (EPS-HEP2017) 513

%\bibitem{uboone}
%B. Jones et al.  \emph{The Status of the MicroBooNE Experiment.~J. Phys.: Conf. Series.} Vol.\textbf{408}. IOP Publishing, 2013,
%doi:10.1088/1742-6596/408/1/012028

%\bibitem{castoreos}
% L. Mascetti et al. \emph{Disk storage at CERN.~J. Phys.: Conf. Series.} Vol.\textbf{664}. IOP Publishing, 2015,
%doi:10.1088/1742-6596/664/4/042035

\bibitem{sam}
R. A. Illingworth \emph{A data handling system for modern and future Fermilab experiments.~J. Phys.: Conf. Series.} Vol.\textbf{513}. IOP Publishing, 2014,
doi:10.1088/1742-6596/513/3/032045

\bibitem{fts}
A. Norman \emph{The Fermilab File Transfer System}.~e-Print: FNAL CD-DocDB-5412

\bibitem{eos_role}
S. Fuess et al. \emph{Design of the protoDUNE raw data management
infrastructure.~J. Phys.: Conf. Series.} Vol.\textbf{898}. IOP Publishing, 2017,
doi:10.1088/1742-6596/898/6/062036

%\bibitem{xrootd}
%L. Bauerdick et al. \emph{Using Xrootd to Federate Regional Storage.~J. Phys.: Conf. Series.} Vol.\textbf{396}. IOP Publishing, 2012,
%doi:10.1088/1742-6596/396/4/042009

\bibitem{panda}
T. Maeno et al. \emph{Overview of ATLAS PanDA Workload Management.~J. Phys.: Conf. Series.} Vol.\textbf{331}. IOP Publishing, 2011,
doi:10.1088/1742-6596/331/7/072024



\bibitem{dirac}
A. Casajus et al.  \emph{DIRAC Pilot Framework and the DIRAC
Workload Management System.~J. Phys.: Conf. Series.} Vol.\textbf{219}. IOP Publishing, 2010,
doi:10.1088/1742-6596/219/6/062049

\bibitem{django}
N. George \emph{Mastering Django: Core. The Complete Guide to Django 1.8 LTS}~ GNW Independent Publishing, ISBN: 099461683X



\end{thebibliography}


\end{document}

% end of file template.tex

%\begin{table}
%\begin{center}
%\caption{\label{table:np04_data_rate}
% protoDUNE-SP readout parameters}
%\ \\
%\begin{tabularx}{0.75\textwidth}{ X  >{\setlength{\hsize}{0.8\hsize}}r}
%\hline
%Detector Parameter & Target \\
%\hline
%TPC channel count & 15,360 \\
%Digitization frequency & 2\,MHz \\
%Nominal electron drift time & 2.2\,ms\\
%Nominal electron drift velocity & 1.6\,mm/$\mu$s\\
%Readout window & 5\,ms \\
%SPS spill time& 4.8\,s\\
%SPS cycle& 22.5\,s\\
%Nominal trigger rate & 25\,Hz \\
%Single readout size (per trigger) & 230.4\,MB \\
%Lossless compression factor (estimated) & 4 \\
%Instantaneous data rate (in-spill) & 1440\,MB/s \\
%Average data rate & 300\,MB/s \\
%Total data recorded (beam + cosmic) & 3\,PB\\
%Buffer to store 3 days worth of data & 300\,TB\\
%\hline
%\end{tabularx}
%\end{center}
%\end{table}

