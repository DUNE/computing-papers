%The Liquid Argon Time Projection Chamber (LArTPC) has the potential to provide exceptional level of detail
%in studies on neutrino interactions - a high prioritory field of Intensity Frontier research. Liquid Argon serves as both
%the target for neutrino interactions and the sensitive medium of the detector, which measures ionization produced by
%the reaction products. The LArTPC has characteristics suitable for precise reconstruction of infividual tracks as well
%as for calorimetric measurements. In order to gain sensitivity to reactions with very small cross-sections, modern
%LArTPC devices are built at a considerable scale, currently in hundreds of tons of instrumented volume of Liquid Argon.
\begin{abstract}
Future experiments such as the Deep Underground Neurtino Experiment (DUNE) will use very large Liquid Argon Projection Chambers
(LArTPC) containing tens of kilotons of cryogenic medium. To be able to utilize sensitive volume  of that size,
%that large while staying within practical limits of power consumption and cost of the front-end electronics,
current design employs arrays of wire electrodes
grouped in readout planes, arranged with a stereo angle. This presents certain challenges for object reconstruction
due to ambiguities inherent in such scheme. We present a novel reconstruction method (named ``Wirecell') inspired by principles
used in tomography, which brings the LArTPC technology closer to its full potential.

\end{abstract}