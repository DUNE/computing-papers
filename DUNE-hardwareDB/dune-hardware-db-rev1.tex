\documentclass[pdftex,12pt,letter]{article}
\usepackage[margin=0.75in]{geometry}
\usepackage{verbatim}
\usepackage{graphicx}
\usepackage{xspace}
\usepackage{cite}
\usepackage{url}
\usepackage[pdftex,pdfpagelabels,bookmarks,hyperindex,hyperfigures]{hyperref}


\title{Functional Requirements for the Hardware Database System in DUNE and Options for its Design}
\date{\today}
\author{M.\,Potekhin, B.\,Viren}


\begin{document}
\maketitle

\begin{abstract}
\noindent  The DUNE experiment requires systems and tools for management, control
and accounting  in the areas of manufacturing, QC, logistics, installation and maintenance.
These systems will include hardware/QC/construction/installation and other databases.
It is important to ensure proper integration, interconnectivity and interoperability of these
databases  as well as interfaces to other systems such as calibration, slow controls etc.
It is also important to ensure the possibility to describe the evolution of various hardware modules
in time in terms of their composition (e.g. circuits replaced) and potentially even their structure;
this implies version control capability.
We propose a hierarchical approach to the description of the apparatus and a state transition
model for the construction process. Accordingly, we consider a few \textit{functional}
requirements to the databases and their interfaces  in the context presented above,
as well as implications and options for the system design.

\end{abstract}

%%%%%%%%%%%%%%%%%%%%%%%%%%%%%%%%%%%%%%%%%%%%%%%%%%%%%

\section{Overview}

The most recent experience in DUNE with systems of this kind comes from the protoDUNE-SP
experiment. The experiment employed several hardware/QC databases which were distinct
for a few elements of the apparatus. For example, the Cold Electronics QC systems were designed specifically
for purposes of testing and quality control of the novel Cold Electronics and as such provided ample capabilities
for storing and analysis of the data collected. A dedicated database was used for managing the CE information.
% Data collected during in the process of testing was archived and made available for user analysis.

Other databases were created (sometimes only as prototypes) to support manufacturing, transport and installation of 
protoDUNE-SP elements such as the PD paddles, APA/CPA and other TPC components. These were distinct from
the CE DB systems and based on the platform commonly referred to as ``FNAL hardware DB'' \cite{hardwareDB}
with separate databases used for different element types. In each such database the
relationships between individual components and modules were described by reference utilizing a
consistent set of primary IDs, specific for the particular DB. No common numbering scheme was used.
Some information (e.g.\,logistics) remained external to the DB and had to be managed separately.

The DUNE experiment represents a vastly larger scale in terms of the component count,
overall complexity and significantly more challenging installation procedures. Optimal execution
of manufacturing, installation, commissioning and operations will require a well thought out design
of the databases in the context presented here. Possible reuse and adaptation of existing technologies
is highly desirable.
% For these reasons it is necessary to define a
% basic set of requirements and design criteria in this work area.

% At the time of writing no consolidated listing of the support tooling, installation equipment, test stands, and other construction infrastructure exists.

\section{The Model}
\subsection{Components and Aggregations}

It is natural to describe the DUNE apparatus in terms of hierarchy of elements e.g.\,model it
as a \textit{graph}. Same applied to its components, for example a FEMB can be modeled as an
\textit{aggregation} which contains various electronics components e.g. ASICS, and can be decsribed as a graph.
This approach was in fact used in the individual DB schemas for protoDUNE-SP elements such as CPA, Photon Detector etc.

Most DUNE components are aggregations themselves, although in graph terms, the leaves of the tree are atomic
by definition (i.e.\,an individual ASIC).  Definining the levels and granularity of the hierarchical structure of components
in DUNE is beyond the scope of this paper, we just note here that this will need to be done in the design phase.
In the following we assume that a \textit{component} may be either atomic or an \textit{aggregation} of other components
subject to the same assumption.

\subsection{Classification}
\label{classification}
To provide consistency to the following set of requirements, we propose to create \textit{``class definitions''} for the components
i.e.\,identify types of components that can be defined in the system utilizing the same set of schemas.

\section{States, Transitions and Component Evolution}

\subsection{Component States and State Transitions}
\label{states}
We propose the following model:

\begin{itemize}

\item Each component can exist in a variety of \textit{states}, and its set of possible states will differ depending on
the component calls. For example, the component can be in states like (``ordered'', ``in transit'', ``received'', ``tested'', ``put in long term storage'',...) etc

\item The chain of events starting from design/ordering/manufacture (or procurement) and then on to QC,
shipping, installation, testing ``in situ'' etc is reflected by the changing state of the component as recorded
in the database

\item The exact sequence of transitions depends on the exact nature (class) of the specific  component; one
critically important case of transition is \textit{aggregation}

\item Each state transition may optionally result in a data product specific to the component and the specific transition.
One example is metrics collected during the QC procedures for CE. The exact nature, size and format of data thus produced
depends on the specific component and transition

\item In the item above, there will be cases when the database will only contain \textit{references} to external data storage
and sources as it is often impractical to store certain data products e.g.\,ROOT files directly in the DB

\item Each state transition may optionally require a record in the audit trail e.g.\,the time stamp, the reason for the event
any comments required in the process and the entity responsible for changing the state and updating the record
(a person and/or institution). Examples already exist in the ``FNAL Hardware DB''

\end{itemize}


\subsection{Requirements related to State Transitions}
\label{history}
\begin{itemize}

\item History: a complete history of state transitions of a component  preserved in the database (where applicable)

\item Timelime: a ``period of validity'' will be recorded for each state, and essentially defined by the timestamps
of transitioning into a particular state and leaving it for the next state

\item If the state transition is due to application of a manufacturing step, a specific test or some other procedure
the nature and the parameters of such procedure must be recorded in the system and be \textit{uniquely identifiable}.

\item The complete set of state transitions (\ref{states}) of a component forms its lifecycle. These may
evolve changing the composition, physical characteristics, location, and associated documentation.
The system must provide adequate means to manage the component lifecycle, including its
testing and QC, \textit{evolving documentation and additional metrics collected after its deployment}

\end{itemize}


\subsection{Aggregate Evolution}
\label{evolution}
Consider the use case where a particular component within an aggregation needs to be replaced,
such as a failed ASIC installed on a board. This process can be modeled in at least two ways:

\begin{itemize}

\item The old aggregation (e.g. the board) is marked as ``retired'' and the audit trail record created; a new aggregation is
created, assigned a new \textit{unique ID} and assumes the place of the old one

\item Or, the aggregation with its existing \textit{unique ID} is preserved but is assigned a new \textit{version number};
again, an audit trail record is created

\end{itemize}

\noindent The concrete choice of solution will depend on the particular component so it needs to be deferred to the
design and implementation stage.

The hierarchical model presented in this paper allows to handle an important use case whereby a component (module)
is completely redesigned in terms of its construction and composition while retaining its functionality and 
place in the overall hierarchy. Same applies to use cases where new aggregations of already existing components
need to be defined for optimal management of the construction process.



\section{Component Location and Ownership}

We will use the concept of the \textit{functional position} introduced in \cite{atlasequipmentdb}, which refers
to a physical position of a component specific to its function and which cannot be moved. We put forth the
following requirements in this area:
\begin{itemize}

\item The  the geographical location and the functional position of the item (if applicable) constitute an integral part of its \textit{state}

\item The current ownership of the item constitute an integral part of its \textit{state}

\item It must be possible to describe the geographical location and functional position of the item with an appropriate
degree of granularity e.g.\,down to a building, room, rack, board, socket etc

\end{itemize}

\section{Connectivity Requirements}
To provide information on expected and actual assembly of components
the following is required:

\begin{itemize}

\item The \textit{expected} or \textit{functional position} connectivity shall be
 modeled for both physical and electrical relationships

\item The \textit{actual} or \textit{as-built} or \textit{as-found} physical and electrical
connectivity shall be modeled in terms that are relative to the expectation

\item The \textit{connectivity} as described in the system will be modeled according to the \textit{state and state transition}
concept as set forth in \ref{states} so it will be subject to the ``history'' requirement as defined in \ref{history}
 
\end{itemize}


\section{Interfaces}

\subsection{Network and Web}

The database will be required to have a variety of interfaces as dictated by its applications e.g.

\begin{itemize}

\item Web interface for operators and managers, which will effectively create a full-fledged information system

\item Other network-based interfaces, including command-line interface which will allow automation (such as scripting) of operations
and integration of various production processes

\end{itemize}



\subsection{Interfaces for the Component Lifecycle Management}
As an example, there are cases like the CE QC and other test data where the metrics thus collected form a part of the
initial calibrations parameters. For this reason, appropriate interfaces must be created to ensure
reliable transfer of such data across the systems. In order to evaluate the performance history and condition of a particular
electronic component it must be possible to get access to subsequent iterations
of the calibration data as well, i.e.\,to provide access to data sources outside of the hardware database itself.

\subsection{Authentication/Authorization}
To properly manage the component lifecycle and a variety of other information presented in the database it will be necessary
to integrate with a role-based authentication/authorization system e.g\, a site SSO.

\subsection{Industrial ID Systems and Mobile Platforms}

The DUNE Hardware Database will have the functionality for integration with barcode and QR scanners to facilitate
all stages of the construction process wherever such devices are available. These do not serve as a substitute to the
unique component ID described above but are complementary to it.

When testing large-scale components of the DUNE experiment and also in the installation stage
it may become necessary to use mobile devices (tablets, scanners etc) in the field to feed information to the database
and also to read data from the database. Such interfaces must be included in the design.

\section{Summary of Functional Requirements}


\subsection{The Unique ID Requirement}

To make it possible to create and evolve the hierarchical description of DUNE along the lines described above
we postulate the following requirement:

\begin{itemize}

\item Every component used in DUNE ---  either atomic or an aggregation --- needs to be \textit{uniquely identifiable} in a way that allows
creation of any type of associations of components

\end{itemize}


\noindent Note that the above requirement is crucial to allow for the eventual evolution of DUNE as it undergoes construction and installation.
New aggregations can thus be created where necessary. The unique identifiers in this context are distinct from the manufacturers' and other ad hoc IDs.
At the same time the unque IDs \textit{are associated} with the manufacturers' and other ad hoc IDs in the database.


\subsection{The Traceability Requirement}

It must be possible, given the ID (and perhaps other attributes or references of a component described in the database)
to identify it in documents such as technical drawings, schematics and any other documentation for QC and installation.
For example, given the component ID it must be possible to identify and retrieve a document describing the relevant QC
procedures. It is implied that the exact provenance (e.g.\,the manufacturer) of each component must be recorded, and
queries based on such attributes must be enabled.

% comment by Jim Stewart:
% I think an additional requirement is traceability. You need from any number to find the drawing where it
% is defined and related documentation for the assembly and QC. If I give you a number you can them tell
% me where to find the drawing/schematic where it is defined and the QC document describing the testing and so on.



\subsection{The Completeness Requirement}

To ensure the functionality of the database we propose the following requirement:
\begin{itemize}
\item each component shall be fully described in the system with level of detail necessary to
ensure successful manufacture, QC, delivery, assembly, installation, commissioning
and operation and maintenance of DUNE, and also document possible failures in any of these steps

\item such information will be specific to each component class (\ref{classification}) in any of its possible states

\end{itemize}

%\noindent Now let us consider these requirements in more detail.

\subsection{Old Requirements}

\begin{itemize}

\item Ability to handle both atomic components and aggregation type components

\item Flexibility in the component definition e.g. the composition, structure and level of aggrregation

\item Unique identification of each component in the context presented above

\item Completeness: all information should be available directly in the database or by reference to a reliable external source
(example: detailed ASIC QC data)

\item Treaceability: provision of references to a variety of technical documentation for a particular component

\item History

\item Connectivity

\item Complete audit trail for each state transition/procedure

\item Auth/Auth

\item Component versioning capability (if such capability is required by choosing the second option in (\ref{evolution})

\item Complete location/ownership information record as part of the state

\item Provisons for integration with industrial automated identification methods such as barcodes, QR codes etc

\item A set of interfaces to allow the creation of a user-friendly information system as well integration of the production, testing, QC,
logistics and installation procedures into the system

\item Interfaces to the Conditions and Calibration databases as well as document handling systems

\end{itemize}


\section{Reusability}

It is foreseen that any design or adaptation of a system that meets the functional requirements listed
above will require a substantial investment of effort and resources. For this reason, we propose to
state a collateral requirement that the system used in DUNE should be experiment-agnostic in
terms of its architecture, and be adaptable to other projects. This will help to assure a more
efficient use of resources and may serve to attract collaboration from other experiments
as a part of a cross-cutting effort.

\section{The Timeline}
Component manufacturing and construction of the DUNE experiment has at least two distinct time
scales that need to be addressed in the context presented here:

\begin{itemize}

\item Near and Medium Term
\begin{itemize}

\item Preparations for the protoDUNE-2 experiment, with manufacturing of components to commence in 2020, and installation scheduled for 2021
\item Early manufacturing of certain components of the DUNE experiment
\end{itemize}

\item Longer term, including manufacturing and installation of the DUNE detector, commissioning and operation
\end{itemize}

\section{The Proposal}
\subsection{Bridging the Gap}

While there is a substantial time pressure to address the needs of DUNE in the near and medium term,
at the time of writing it seems unlikely that a system capable of addressing the requirements
presented in this paper can be created (or adopted from another experiment) on such short timescale.
We propose to mitigate this in the following manner:

\begin{itemize}

\item Survey and catalog the interim systems (e.g.\,existing and in use at participating sites)
which are likely to be used in near and medium term

\item Identify additional attributes (if any) to be added to object models in these systems
and modest changes in functionality to allow for eventual migration to the final system

\item Define formats for data export and ingestion necessary for such migration as well as future
I/O and data exchange protocols

\end{itemize}

\subsection{Action Items}
In addition to interim measures described above we propose the following action items:
\begin{itemize}

\item conduct a comprehensive survey of the DUNE Consortia and computing groups
in order to agree on a common set of requirements for the database in the context presented
in this paper

\item evaluate a few different design options, such as summarized in the
next section (\ref{options}). The ultimate aim of this effort is to preferably identify a possible
existing solution or perhaps a new deisgn satisfying the requirements of the Collaboration at large. 

\end{itemize}

\noindent The key motivations are:

\begin{itemize}

\item to avoid duplication of effort and meet the requirements to the greatest extent possible

\item to prevent divergent and/or incompatible designs in each sybsystem area and avoid operational risks associated with that

\item to provide a system-wide view of the production, testing, QC, logistics and installation process necessary
for efficiency of management, identification of problem areas and bottlenecks and minimizing errors in all stages
of the DUNE construction

\end{itemize}

% As one of the first steps in the design process, we propose to survey the systems used by individual
%teams and groups reponsible for R\&D in  DUNE up to this point, and including the more recent experience in
%the construction and commissioning of the proroDUNE-SP experiment. 

\section{Options}
\label{options}

\subsection{Reuse}
Possible reuse of components and/or existing systems should be considered whenever possible,
with adaptations as necessary to conform to the functional requirements outlined above. We consider
two candidates below, without necessarily limiting ourselves to these two.

\subsubsection{The FNAL Hardware DB}

The current version of the FNAL Hardware DB \cite{hardwareDB} is a collection of tools aimed at facilitating the creation,
management and use of the hardware databases required by the groups working on manufacturing and QC of 
various detector
components. The system was used in building two subsystem of the protoDUNE-SP experiment.

\begin{itemize}

\item \textbf{Pros}
\begin{itemize}
\item A well understood system proven in previous experiments, with good support and expertise at FNAL
\item Supports many of the requirements listed above
\end{itemize}

\item \textbf{Cons}
\begin{itemize}
\item A rigid schema not well suited for evolution (e.g.\,QC procedures and logistics essentially hardcoded in the schemas
created for individual components)
\item Still lacking many features necessary to meet the DUNE requirements, lack of integration with other components
modeled in the same framework and with external systems
\item Current version of the Web UI is not yet well suited to support the functionality to meet the requirements discussed here
\end{itemize}

\item \textbf{Reuse strategy}
\begin{itemize}
\item Using experience with the system, upgrade it by introducing a Web framework allowing for OO-based modeling and schema evolution (cf.\,ORM)
\item Implement integration with external systems
\item Update the Web UI and other clients
\end{itemize}


\end{itemize}

\subsubsection{ATLAS Equipment Database}

An introduction to the ATLAS Equipment DB is presented in \cite{atlasequipmentdb}.

\begin{itemize}

\item  \textbf{Pros}
\begin{itemize}
\item A full featured system covering many complex use cases
\item Likely supports most and perhaps all of the DUNE requirements
\item Proven during construction, installation and operation of a very complex detector
\item Integration of a number of separate databases (using the Glance system, \cite{glance})
\end{itemize}

\item \textbf{Cons}
\begin{itemize}
\item Substantially based on a proprietory commercial system (Infor EAM \cite{infor}, formerly d7i), entailing the licensing and lifecycle issues
\item Has many hooks into the vast CERN information infrastructure which would be hard to disentangle in the reuse scenario
\item Considerable amount of work was required to tool the database for specific ATLAS needs, this part won't be immediately reusable in any case
\end{itemize}

\end{itemize}

\subsection{New Development}
At present, creation of a new hardware database for DUNE is also on the table as a design option. Discussion
of the possible design directions and cost/effort estimations relative to adoption of an existing solution
are deferred till a later time.



% \clearpage

\begin{thebibliography}{1}

\bibitem{hardwareDB}
{\textit{Hardware DB Wiki} \url{https://cdcvs.fnal.gov/redmine/projects/hardwaredb/wiki}}

\bibitem{atlasequipmentdb}
{\textit{\mbox{Management of Equipment Databases at CERN for the ATLAS Experiment}}
\mbox{DOI: 10.1142/9789812819093\_0129}
}

\bibitem{glance}
{\textit{\mbox{Glance project: a database retrieval mechanism for the ATLAS detector}}
\mbox{DOI: 10.1088/1742-6596/119/4/042020}
}

\bibitem{infor}
{\textit{Infor EAM} \url{https://www.infor.com/products/eam}}

\end{thebibliography}


\end{document}
