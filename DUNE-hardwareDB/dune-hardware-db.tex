\documentclass[pdftex,12pt,letter]{article}
\usepackage[margin=0.75in]{geometry}
\usepackage{verbatim}
\usepackage{graphicx}
\usepackage{xspace}
\usepackage{cite}
\usepackage{url}
\usepackage[pdftex,pdfpagelabels,bookmarks,hyperindex,hyperfigures]{hyperref}


\title{A Brief Review of the Requirements for the Hardware/Construction Database Systems in DUNE}
\date{\today}
\author{M.\,Potekhin, B.\,Viren}


\begin{document}
\maketitle

\begin{abstract}
\noindent  Due to its complexity DUNE requires systems and tools for control, accounting and
management in the area of manufacturing, QC, logistics, installation and maintenance. These systems
will involve databases which include hardware/QC/construction/installation DBs,
``Parts Breakdown Structure`` and others. The challenge is to ensure proper
interconnectivity and interoperability of these databases with others such as calibration,
slow controls etc which is necessary for commissioning and continued
operations. We consider minimal reasonable requirements to the design of the databases
and their interfaces. Technology downselect is outside of the scope of this document.

\end{abstract}

%%%%%%%%%%%%%%%%%%%%%%%%%%%%%%%%%%%%%%%%%%%%%%%%%%%%%

\section{Overview}

The most recent experience in DUNE with systems of this kind comes from the protoDUNE-SP experiment.
The experiment employed hardware/QC databases which were distinct in both
design and implementation for each major element of construction and installation. For example, the CE QC
systems were designed specifically for purposes of testing and quality control of the Cold Electronics,
and were distinct from databases used in manufacturing of the FEMB, APAs/CPAs, cabling database etc.
The DUNE experiment represents a vastly larger scale in terms of the component count and
overall complexity and will require well-coordinated design of such databases. For that reasons,
it is necessary to define a base set of requirements and design criteria in this work area.

\section{Basic Assumptions}
\subsection{Components and Associations}

Consider the following basic assumptions regarding information about components and their relationships:

\begin{itemize}

\item Every hardware component used in DUNE needs to be uniquely identifiable, described in the
system and accompanied by information with content specific to the component and its state, and level
of detail necessary to ensure successful manufacture, QC, delivery, assembly, installation, commissioning
and operation and maintenance of DUNE

\item Likewise, \textit{associations} of the components need to be uniquely identifiable and described
in the system in ways specific to each situation; ``association'' may mean an assembly or aggregation of
components in a hierarchical manner e.g.\,electronic components installed on a FEMB, or in terms of
point-to-point connections between components or subsystems

\end{itemize}

\noindent Now let us consider these assumptions in more detail.

\subsection{Components}
Assumptions:

\begin{itemize}

\item Each component can exist in a variety of states. The chain of events starting from design/ordering/manufacture and
then on to QC, shipping, installation, testing ``in citu'' etc is reflected by the changing state of the component as recorded
in the database

\item The exact sequence of transitions depends on the exact nature of the specific  component

\item Each state transition may optionally result in a data product specific to the component and the specific transition;
one example is metrics collected during the QC procedures for CE; the exact nature, size and format of data thus produced
depends on the specific component and transition

\item Each state transition may optionally require a record in the audit trail, i.e.\,reason for the event and the entity
responsible for changing the record (a person and/or institution)

\end{itemize}



\clearpage

\begin{thebibliography}{1}

\bibitem{test}
{\textit{A test} \url{https://arxiv.org/abs/1712.06982}}

\end{thebibliography}


\end{document}
